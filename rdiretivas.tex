% Uso de pacotes e configurações
% ------------------------------------------------------------------------------

% Atenção: A alteração da ordem dos pacotes pode causar problemas.

\usepackage[brazil]{babel} % Opções de typesetting para pt_BR.
\usepackage[utf8]{inputenc} % Input encoding deve ser UTF-8
\usepackage[T1]{fontenc} % Torna possível extrair texto do PDF com acentuação correta.

\usepackage{lscape} % O ambiente lscape permite girar imagens 90 graus.
\usepackage{graphicx} % Define uma interface melhor para inserir imagens.
\usepackage{array, longtable}

% Faz com que as referências fiquem clicáveis e o PDF contenha bookmarks. 
% É possível mudar as cores dos links.
\usepackage[pdftex, bookmarks=true, colorlinks=false]{hyperref}

\usepackage{amsmath, amsthm, amssymb}
\usepackage{caption} % Permite colocar quebras de linha em captions com \newline
\usepackage{ifthen}
\usepackage{pdfpages}

% Referências alfabéticas.
\usepackage[alf]{abntcite}

\usepackage{trimspaces}

% Pacotes próprios.
\usepackage{rtese}
\usepackage{rdefs}
\usepackage{rsiglas}
\usepackage{rcitacaole}
\usepackage{rimg}



